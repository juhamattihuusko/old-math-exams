\documentclass[12pt]{article}
\usepackage{graphicx} % Required for inserting images
\usepackage{amsmath,amssymb,amsthm,amsfonts}
\usepackage{xcolor}
\usepackage{tasks}
\usepackage{enumitem}
\usepackage[margin=2cm]{geometry}

\title{OAMK}
\author{Juha-Matti Huusko}
\date{August 2023}

\newcommand{\ratkaisu}[1]{{\color{blue}\quad\textrm{Solution } #1}}

%\renewcommand{\ratkaisu}[1]{}

\begin{document}
\thispagestyle{empty}

\section*{Additional physics problems for your self study in case you are interested}
%\section*{Uusintakoe 17.11.2023}
%\section*{Loppukoe 20.10.2023}

On 25.2.2025, some students were asking for more problems in physics.\\

\noindent Here are some problems which you can practice if you like. This file was compiled in the same day and can contain misprints.

\begin{enumerate}
\item A mass $m=2~kg$ falls in vacuum. Only force is the gravitation of Earth. The initial position $y(0)=0$ and the initial velocity $y'(0)=0$. Which differential equation does the mass satisfy?  How long distance will the mass travel in 3~s?

\item A mass $m$ rests on a frictionless horizontal rail. The mass is attached to a wall with a spring (spring constant $k$). Let $y(t)$ be the displacement of the mass from the equilibrium point. Assume that $y(0)=0.3~m$ and $y'(0)=0~m/s$. Which differential equation does the mass satisfy? At which time does the mass, for the first time, pass the equilibrium point?

\item Consider a radioactive isotope. Denote the number of particles of the isotope with $y(t)$. Let $y(0)=100$ and $y(3)=50$. Which differential equation does the isotope satisfy? For which value of $t$ we have $y(t)=12$?

\item In a pendulum, a mass $m=1~kg$ is hanged in a rope of length $L=1~m$. Let $\theta(t)$ be the displacement angle from the neutral position. Assume that $\theta(0)=5^\circ$ and $\theta'(0)=0$. Which differential equation does $\theta(t)$ satisfy? In which time does the pendulum complete a full swing back and forth?
\end{enumerate}

\end{document}

\newpage
\section*{Formulas}
\subsection*{Differentiation and integration}

$$
\begin{array}{rl|rl}
\textbf{Differentiation} && \textbf{Integration}&\\[2mm]
Dx^n&=nx^{n-1}     \qquad\qquad&\qquad\qquad\int x^ndx&=\frac{x^{n+1}}{n+1}+C \\[2mm]
De^x&=e^x &\int e^xdx&=e^x+C\\[2mm]
Db^x&=b^x\ln(b) & \int b^xdx&=\frac{b^x}{\ln(b)}\\[2mm]
D\ln(x)&=\frac{1}{x} &&\\[2mm]
D\ln|x|&=\frac{1}{x} &\int\frac{1}{x}dx&=\ln|x|+C\\[2mm]
D\log_a(x)&=\frac{1}{x\ln(a)} &&\\[2mm]
D\log_a|x|&=\frac{1}{x\ln(a)} &&\\[2mm]
D\sin(x)&=\cos(x)   &\int\cos(x)dx&=\sin(x)+C\\[2mm]
D\cos(x)&=-\sin(x)  &\int\sin(x)dx&=-\cos(x)+C\\[2mm]
D\tan(x)&=1+\tan^2(x) \qquad&\qquad\int 1+\tan^2(x)dx&=\tan(x)+C\\[2mm]

Dx\ln(x)-x&=\ln(x) & \int\ln(x)dx&=x\ln(x)-x+C\\[10mm]

D\arcsin(x)&=\frac{1}{\sqrt{1-x^2}} & \int\frac{1}{\sqrt{1-x^2}}&=\arcsin(x)+C\\
D\arccos(x)&=\frac{1}{-\sqrt{1-x^2}} & \int\frac{1}{-\sqrt{1-x^2}}&=\arccos(x)+C\\
D\arctan(x)&=\frac{1}{1+x^2} & \int\frac{1}{1+x^2}&=\arctan(x)+C\\

D\sinh(x)&=\cosh(x) &&\\
D\cosh(x)&=\sinh(x) &&\\
D\tanh(x)&=\frac{1}{\cosh^2(x)} &&\\
\end{array}  
$$
\vspace{1cm}
$$
\begin{array}{rl|rl}
\textbf{Differentiation} && \textbf{Integration}&\\[2mm]
D f(g(x))&=f'(g(x))g'(x) & \int f(g(x))g'(x)dx&=f(g(x))+C\\[2mm]
\textrm{Special cases} &&&\\
D\ln(g(x))&=\frac{g'(x)}{g(x)} & \int \frac{g'(x)}{g(x)}dx&=\ln(g(x))+C\\[2mm]
D e^{g(x)}&=e^{g(x)}g'(x) & \int g'(x)e^{g(x)}dx&=e^{g(x)}+C\\[10mm]
D fg&=f'g+fg'& \int f'g dx&=fg-\int fg'dx\\[2mm]
D (f/g)&=(gf'-fg')/g^2 &&\\[2mm]
\end{array}  
$$

\newpage

\subsection*{Physics}

\subsubsection*{Newtonian mechanics}

Some forces
\begin{itemize}
\item $G=mg$ (gravitation)
\item $F=kx$ (spring)
\item $F=av^2$ (air resistance)
\end{itemize}
Differential equation is given by Newton's second law
$$
F_{\textrm{total}}=mx''.
$$

\subsubsection*{Lagrangian mechanics}
Lagrange function is the difference of kinetic energy and potential energy. As a formula, $L=T-U$ or $L=K-P$
\begin{itemize}
\item Object in free fall $L=\frac12 my''-mgy$
\item Pendulum $L=\frac12 mL^2\theta'(t)-mgL(1-\cos(\theta(t)))$
\item Mass and spring $L=\frac12 mx'(t)^2-\frac12 kx(t)^2$
\end{itemize}
Differential equation is given by Euler-Lagrange equation
$$
\frac{d}{dt}\left(\frac{\partial L}{\partial x'}\right)-\frac{\partial L}{\partial x}=0.
$$

\subsubsection*{Other differential equations}

\begin{itemize}
\item Radioactive decay $N'(t)=-aN(t)$
\item Newton's law of cooling $T'(t)=-k(T(t)-T_a)$
\end{itemize}

\newpage
\subsection*{Differential equations}
\subsubsection*{Second order linear ODE with constant coefficients}

\begin{itemize}
\item ODE $y''+by'+cy=0$
\item Characteristic equation $r^2+br+c=0$
\end{itemize}
Cases
\begin{itemize}
\item $r_1,r_2\in\mathbb{R}$ solution
$$
y(x)=A\exp(r_1x)+B\exp(r_2x)
$$
\item $r_1=r_2=r$ solution
$$
y(x)=A\exp(rx)+Bx\exp(rx)
$$
\item $r_1=a+bi$ solution
$$
y(x)=\exp(ax)(A\cos(bx)+B\sin(bx))
$$
\end{itemize}

\subsubsection*{Integrable ODE}
The solution of
$$
y'=q(x)
$$
is $y(x)=\int q(x)dx$

\subsubsection*{Separable ODE}
If you can arrange the equation as $$a(y)dy=b(x)dx,$$
then you can integrate to obtain $$\int a(y)dy=\int b(x)dx.$$

\subsubsection*{First order linear ODE}
The solution of
$$
y'+p(x)y=q(x)
$$
is
$$
y(x)=\frac{C}{\mu(x)}+\frac{1}{\mu(x)}\int \mu(x)q(x)dx,\quad\textrm{where}\quad
\mu(x)=e^{\int p(x)dx}.
$$

\newpage
\subsection*{Fourier series}

If $f$ is periodic with period $2\pi$ and $f$, $f'$ and $f''$ are piece-wise continuous, then
$$
f(x)
=\frac{a_0}{2}
+\sum_{n=1}^\infty a_n\cos(nx)+b_n\sin(nx),
$$
where
\begin{equation*}
\begin{split}
a_0&=\frac{1}{\pi}\int_{-\pi}^{\pi}
f(x)dx\\
a_n&=\frac{1}{\pi}\int_{-\pi}^{\pi}
f(x)\cos(nx)dx\\
b_n&=\frac{1}{\pi}\int_{-\pi}^{\pi}
f(x)\sin(nx)dx\\
\end{split}
\end{equation*}
Moreover, if $f$ is odd, that is, $f(-x)=-f(x)$,
then
$$
f(x)
=\sum_{n=1}^\infty b_n\sin(nx),
$$
and if $f$ is even, that is, $f(-x)=f(x)$, then
$$
f(x)
=\frac{a_0}{2}
+\sum_{n=1}^\infty a_n\cos(nx)
$$

\subsection*{Discrete Fourier transform / FFT}

Transform and inverse transform
$$
\begin{cases}
y_0&=x_0+x_1\\
y_1&=x_0-x_1
\end{cases},\quad
\begin{cases}
y_0&=\frac{1}{2}(x_0+x_1)\\
y_1&=\frac{1}{2}(x_0-x_1)
\end{cases}
$$
Transform and inverse transform
$$
\begin{cases}
y_0&=x_0+x_1+x_2+x_3\\
y_1&=x_0-ix_1-x_2+ix_3\\
y_2&=x_0-x_1+x_2-x_3\\
y_3&=x_0+ix_1-x_2-ix_3\\
\end{cases},\quad
\begin{cases}
y_0&=\frac{1}{4}(x_0+x_1+x_2+x_3)\\
y_1&=\frac{1}{4}(x_0+ix_1-x_2-ix_3)\\
y_2&=\frac{1}{4}(x_0-x_1+x_2-x_3)\\
y_3&=\frac{1}{4}(x_0-ix_1-x_2+ix_3)\\
\end{cases}
$$


\end{document}
