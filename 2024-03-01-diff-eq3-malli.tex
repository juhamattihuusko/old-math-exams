\documentclass[12pt]{article}
\usepackage{graphicx} % Required for inserting images
\usepackage{amsmath,amssymb,amsthm,amsfonts}
\usepackage{xcolor}
\usepackage{tasks}
\usepackage{enumitem}
\usepackage[margin=2cm]{geometry}

\title{OAMK}
\author{Juha-Matti Huusko}
\date{August 2023}

\newcommand{\ratkaisu}[1]{{\color{blue}\quad\textrm{Solution } #1}}

%\renewcommand{\ratkaisu}[1]{}

\begin{document}
\thispagestyle{empty}

\section*{Practice exam}
%\section*{Uusintakoe 17.11.2023}
%\section*{Loppukoe 20.10.2023}
\section*{Applied Mathematics and Physics in Programming ID00CS50-3003}
\subsection*{Mathematics}

The exam will probably have this kind of questions. The difficulty level can be about like this. The questions can still be modified.

\begin{enumerate}
\item Concepts about differential equations.
\begin{enumerate}


\item Which one of the following equations has order $3$?
$$
y'+y^3=x,\quad 4
y''+xy'+\sin(x)y=3,\quad
\boxed{y'''+x^2y=\frac{1}{x}}.
$$%\ratkaisu{$y'''+x^2y=\frac{1}{x}$}
\item Which one of the following equations is linear?
$$
y'+2xy=\frac{1}{2x+3y},\quad 
\boxed{y'+\sin(x)y=e^x},\quad 
y''+yy'+2xy=3.
$$%\ratkaisu{$y'+\sin(x)y=e^x$}
\item Which one of the following equations is homogeneous?
$$
y'+y+x-3=0,\quad
\boxed{y'+\sin(x)y=0},\quad
y''+y'=x-y.
$$
%\ratkaisu{$y'+\sin(x)y=0$}
\end{enumerate}
\item Show that $y=\frac{1}{1-x}$ is a particular solution for $y'=y^2$.\ratkaisu{The derivative is
$$
y'=\frac{d}{dx}(1-x)^{-1}=-1\cdot (1-x)^{-1-1}\frac{d}{dx}(1-x)=-1\cdot (1-x)^{-2}\cdot(-1)=\frac{1}{(1-x)^2}=y^2
$$}

\item The general solution of $y'=4x^2$ is $y=\frac{4}{3}x^3+C$, where $C$ is any constant. Which particular solution passes through the point $(-3,-30)$ (that is, satisfies $x=-3$, $y=-30$)?
\ratkaisu{We have $y(x)=y=\frac{4}{3}x^3+C.$ Set $x=-3$ and $y(-3)=-30$ to obtain
$$
-30=\frac{4}{3}(-3)^3+C,
$$
that is
$$
-30=-36+C.
$$
We have $C=6$. The desired solution is $y(x)=y=\frac{4}{3}x^3+6$.
}
\item Solve
$$
y'+\frac{y}{x}=x^2
$$
by following the instructions.
\begin{enumerate}
\item Identify $p(x)$ and $q(x)$.\ratkaisu{$p(x)=\frac{1}{x}$ and $q(x)=x^2$}
\item Calculate $\int p(x)dx$. Don't add a constant $C$ yet.\ratkaisu{$\int p(x)dx=\ln(x)$}
\item Simplify $\mu(x)=e^{\int p(x)dx}$ and $\frac{1}{\mu(x)}$\ratkaisu{$\mu(x)=x$ and $\frac{1}{\mu(x)}=\frac{1}{x}$}
\item Calculate $\int \mu(x)q(x)dx$.
\ratkaisu{$\frac{x^4}{4}$}
\item The solution is $y(x)=\frac{C}{\mu(x)}+\frac{1}{\mu(x)}\int \mu(x)q(x)dx$.
\ratkaisu{$y(x)=\frac{C}{x}+\frac{x^3}{4}$}
\end{enumerate}
\item Let $f(x)=\frac{1}{4}x^2$ for $-\pi\leq x\leq \pi$ and let $f(x)$ be periodic with period $2\pi$. It's Fourier series is
\begin{equation}
\label{eq:f}
f(x)=\frac{\pi^2}{12}
+\sum_{n=1}^\infty (-1)^n\frac{1}{n^2}\cos(nx).
\end{equation}
(a) Find the Fourier series of the $2\pi$ periodic function $g(x)$ for which 
$$
g(x)=\frac{x}{2},\quad\textrm{when}\quad -\pi\leq x\leq\pi.
$$
(b) Is $f(x)$ or $g(x)$ odd?\\
(c) Is $f(x)$ or $g(x)$ even?\\
\ratkaisu{We have $Df(x)=g(x)$. Therefore, differentiate equation~\eqref{eq:f} on both sides to obtain
$$
g(x)=-\sum_{n=1}^\infty (-1)^n\frac{1}{n}\sin(nx).
$$
}

\item Calculate by hand the discrete Fourier transform of $[2,3]$. In other words, calculate by hand $\texttt{fft}([2,3])$.
\ratkaisu{$\texttt{fft}([2,3])=[5,-1]$}
\end{enumerate}

\newpage
\section*{Formulas}
\subsection*{Differentiation and integration}

$$
\begin{array}{rl|rl}
\textbf{Differentiation} && \textbf{Integration}&\\[2mm]
Dx^n&=nx^{n-1}     \qquad\qquad&\qquad\qquad\int x^ndx&=\frac{x^{n+1}}{n+1}+C \\[2mm]
De^x&=e^x &\int e^xdx&=e^x+C\\[2mm]
Db^x&=b^x\ln(b) & \int b^xdx&=\frac{b^x}{\ln(b)}\\[2mm]
D\ln(x)&=\frac{1}{x} &&\\[2mm]
D\ln|x|&=\frac{1}{x} &\int\frac{1}{x}dx&=\ln|x|+C\\[2mm]
D\log_a(x)&=\frac{1}{x\ln(a)} &&\\[2mm]
D\log_a|x|&=\frac{1}{x\ln(a)} &&\\[2mm]
D\sin(x)&=\cos(x)   &\int\cos(x)dx&=\sin(x)+C\\[2mm]
D\cos(x)&=-\sin(x)  &\int\sin(x)dx&=-\cos(x)+C\\[2mm]
D\tan(x)&=1+\tan^2(x) \qquad&\qquad\int 1+\tan^2(x)dx&=\tan(x)+C\\[2mm]

Dx\ln(x)-x&=\ln(x) & \int\ln(x)dx&=x\ln(x)-x+C\\[10mm]

D\arcsin(x)&=\frac{1}{\sqrt{1-x^2}} & \int\frac{1}{\sqrt{1-x^2}}&=\arcsin(x)+C\\
D\arccos(x)&=\frac{1}{-\sqrt{1-x^2}} & \int\frac{1}{-\sqrt{1-x^2}}&=\arccos(x)+C\\
D\arctan(x)&=\frac{1}{1+x^2} & \int\frac{1}{1+x^2}&=\arctan(x)+C\\

D\sinh(x)&=\cosh(x) &&\\
D\cosh(x)&=\sinh(x) &&\\
D\tanh(x)&=\frac{1}{\cosh^2(x)} &&\\
\end{array}  
$$
\vspace{1cm}
$$
\begin{array}{rl|rl}
\textbf{Differentiation} && \textbf{Integration}&\\[2mm]
D f(g(x))&=f'(g(x))g'(x) & \int f(g(x))g'(x)dx&=f(g(x))+C\\[2mm]
\textrm{Special cases} &&&\\
D\ln(g(x))&=\frac{g'(x)}{g(x)} & \int \frac{g'(x)}{g(x)}dx&=\ln(g(x))+C\\[2mm]
D e^{g(x)}&=e^{g(x)}g'(x) & \int g'(x)e^{g(x)}dx&=e^{g(x)}+C\\[10mm]
D fg&=f'g+fg'& \int f'g dx&=fg-\int fg'dx\\[2mm]
D (f/g)&=(gf'-fg')/g^2 &&\\[2mm]
\end{array}  
$$

\newpage
\subsection*{Solution formula}

The solution of
$$
y'+p(x)y=q(x)
$$
is
$$
y(x)=\frac{C}{\mu(x)}+\frac{1}{\mu(x)}\int \mu(x)q(x)dx,\quad\textrm{where}\quad
\mu(x)=e^{\int p(x)dx}.
$$

\subsection*{Fourier series}

If $f$ is periodic with period $2\pi$ and $f$, $f'$ and $f''$ are piece-wise continuous, then
$$
f(x)
=\frac{a_0}{2}
+\sum_{n=1}^\infty a_n\cos(nx)+b_n\sin(nx),
$$
where
\begin{equation*}
\begin{split}
a_0&=\frac{1}{\pi}\int_{-\pi}^{\pi}
f(x)dx\\
a_n&=\frac{1}{\pi}\int_{-\pi}^{\pi}
f(x)\cos(nx)dx\\
b_n&=\frac{1}{\pi}\int_{-\pi}^{\pi}
f(x)\sin(nx)dx\\
\end{split}
\end{equation*}
Moreover, if $f$ is odd, that is, $f(-x)=-f(x)$,
then
$$
f(x)
=\sum_{n=1}^\infty b_n\sin(nx),
$$
and if $f$ is even, that is, $f(-x)=f(x)$, then
$$
f(x)
=\frac{a_0}{2}
+\sum_{n=1}^\infty a_n\cos(nx)
$$

\subsection*{Discrete Fourier transform / FFT}

Transform and inverse transform
$$
\begin{cases}
y_0&=x_0+x_1\\
y_1&=x_0-x_1
\end{cases},\quad
\begin{cases}
y_0&=\frac{1}{2}(x_0+x_1)\\
y_1&=\frac{1}{2}(x_0-x_1)
\end{cases}
$$
Transform and inverse transform
$$
\begin{cases}
y_0&=x_0+x_1+x_2+x_3\\
y_1&=x_0-ix_1-x_2+ix_3\\
y_2&=x_0-x_1+x_2-x_3\\
y_3&=x_0+ix_1-x_2-ix_3\\
\end{cases},\quad
\begin{cases}
y_0&=\frac{1}{4}(x_0+x_1+x_2+x_3)\\
y_1&=\frac{1}{4}(x_0+ix_1-x_2-ix_3)\\
y_2&=\frac{1}{4}(x_0-x_1+x_2-x_3)\\
y_3&=\frac{1}{4}(x_0-ix_1-x_2+ix_3)\\
\end{cases}
$$


\end{document}
