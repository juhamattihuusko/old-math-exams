\documentclass[12pt]{article}
\usepackage{graphicx} % Required for inserting images
\usepackage{amsmath,amssymb,amsthm,amsfonts}
\usepackage{xcolor}
\usepackage{tasks}
%\usepackage{enumitem}
\usepackage[margin=2cm]{geometry}
\usepackage{tkz-euclide}

\usepackage[utf8]{inputenc}
\usepackage[T1]{fontenc}
\usepackage{amsmath}
\usepackage{amsfonts}
\usepackage{amssymb}
\usepackage[version=4]{mhchem}
\usepackage{stmaryrd}
\usepackage{enumerate}
\usepackage{multicol}
\usepackage{xcolor}
\usepackage{graphicx}
\usepackage{ulem}
\usepackage{cancel}
\usepackage{tikz}
\usepackage{tkz-euclide}
\usepackage[finnish]{babel}

%\usepackage[style=alphabetic,]{biblatex}

%\usepackage[margin=2cm]{geometry}

\newcommand{\brac}[1]{\left(#1\right)}
\newcommand{\sqbrac}[1]{\left[#1\right]}
\newcommand{\set}[1]{\left\{#1\right\}}

\newcommand{\dd}[0]{\mathrm{d}}
\newcommand{\dx}[0]{\mathrm{d}x}

\newcommand{\hatu}{\hat{u}}
\newcommand{\hatv}{\hat{v}}
\newcommand{\hatw}{\hat{w}}
\newcommand{\hatn}{\hat{n}}

\newcommand{\vu}{\overline{u}}
\newcommand{\vv}{\overline{v}}
\newcommand{\vw}{\overline{w}}
\newcommand{\vp}{\overline{p}}
\newcommand{\vn}{\overline{n}}

\newcommand{\va}{\overline{a}}
\newcommand{\vb}{\overline{b}}
\newcommand{\vc}{\overline{c}}
\newcommand{\vd}{\overline{d}}


\newcommand{\vi}{\hat{\imath}}
\newcommand{\vj}{\hat{\jmath}}
\newcommand{\vk}{\hat{k}}

\newcommand{\ratkaisu}[1]{\hfill{\color{blue}\quad\textrm{Ratkaisu: } #1}}

\newcommand{\ratkaisuu}[1]{{\color{blue}\textrm{Ratkaisu: } #1}}

\newcommand{\kaava}[1]{{\color{green!50!black}#1}}

%\renewcommand{\ratkaisu}[1]{}
%\renewcommand{\ratkaisuu}[1]{}
%\renewcommand{\kaava}[1]{}

\newcommand{\vihje}[1]{{\color{red}Vihje. #1}}
\newcommand{\extra}[0]{\textbf{Extra.}~}

\title{OAMK}
\author{Juha-Matti Huusko}
\date{August 2023}

\renewcommand{\ratkaisu}[1]{{\color{blue}\quad\textrm{Ratkaisu: } #1}}

\renewcommand{\ratkaisu}[1]{}

\begin{document}
\thispagestyle{empty}

\section*{Fysiikan koe}
\subsubsection*{Soveltava fysiikka ja matematiikka mittauksissa IN00CS87-3010}

Laskemalla kolme tehtävää voit saada täydet pisteet.\\ Jos lasket neljä tehtävää, arvioinnissa otetaan huomioon kolme parasta.

\begin{enumerate}
\item Jos mittaustulokseksi on annettu $15,14m/s\pm 2~\%$ miltä väliltä voit odottaa todellisen arvon löytyvän?
\item Oletetaan, että sinulla on suorakulmion mallinen teräslevy. Mittaat sen pituudeksi viivottimella \textbf{8.5} tuumaa, leveydeksi viivottimella \textbf{2.8} tuumaa ja paksuudeksi mikrometriruuvilla 5.98mm. Valmistajan antamien tietojen mukaan teräksen tiheys on  $0.284 lbs/in^3$. Huomaa, että mittaustulosten epätarkkuudet on nyt ilmaistu merkitsevien numeroiden avulla. Laske ja ilmaise vastaukset seuraaviin kysymyksiin oikealla tarkkuudella SI-yksiköissä.
\begin{enumerate}
\item Ilmaise annetut lukuarvot SI-yksiköissä.
\item Mikä on levyn suurimman tahkon pinta-ala? (Yksikkönä $m^2$)
%\item Mikä on suorakulmion piiri eli kaikkien sivujen summa? (Yksikkönä $m$.)
\item Mikä on teräslevyn tilavuus? (Yksikkönä $m^3$)
\item Mikä on teräslevyn massa?
(massa=tilavuus*tiheys, yksikkönä $kg$)
\end{enumerate}
\item \textbf{Kahdensadan} metrin pikajuoksussa juoksija ($m=60kg$) lähtee liikkeelle levosta ja kiihdyttää ensin vakiokiihtyvyydellä 2 sekunnin ajan kunnes hänen nopeutensa on 10m/s. Loput matkasta hän juoksee tällä saavuttamallaan vakionopeudella.
\begin{enumerate}
\item Mikä on juoksijan kiihtyvyys ensimmäisen kahden sekunnin aikana?
\item Mikä on juoksijan keskinopeus kiihdytysvaiheen aikana?
\item Kuinka pitkän matkan juoksija etenee kiihdytysvaiheen aikana?
\item Mikä on juoksijan aika maalissa?
%\item Kiihdytysvaiheen aikana, mikä on keskimääräinen juoksijan kenkien juoksurataan kohdistama \textbf{vaakasuora} voima?
\end{enumerate}
\item Ammutaan tykillä tasaisella maalla maanpinnan tasolta $60^\circ$
kulmaan alkunopeudella \textbf{50m/s}. Unohdetaan ilmanvastus. Voit käyttää putoamiskiihtyvyydelle arvoa $g=10m/s^2$.
\begin{enumerate}
\item Mitkä ovat alussa nopeuden vaaka- ja pystysuorat komponentit $v_{0x}$ ja $v_0y$?
\item Kauanko kestää, että ammus on saavuttanut maksimikorkeutensa?
\item Mikä on ammuksen maksimikorkeus?
\item Kuinka pitkälle ammus lentää?
\end{enumerate}
\end{enumerate}

\section*{Kaavoja}

\subsection*{Yksikkömuunnokset}
\begin{itemize}
\item 1 tuuma = 2.54cm
\item 1 pauna = 1 lbs = 453.6 g
\end{itemize}

\subsection*{Newtonin lait}
\begin{enumerate}
\item $F_{\textrm{kokonais}}=0 \quad\Rightarrow\quad v=\textrm{vakio}$
\item $F_{\textrm{kokonais}}=ma$
\item $F_1=-F_2$
\end{enumerate}

\subsection*{Vino heittoliike}

Vaakasuunnassa tasainen liike
$$
\begin{cases}
a_x(t)=x''(t)&=0\\
v_x(t)=x'(t)&=v_{0x}\\
x(t)&=v_{0x}t
\end{cases}
$$

Pystysuunnassa tasaisesti kiihtyvä liike
$$
\begin{cases}
a_y(t)=y''(t)&=-g\\
v_y(t)=y'(t)&=v_{0y}-gt\\
y(t)&=y_0+v_{0y}t-\frac12 gt^2
\end{cases}
$$

\subsection*{Voimia}

\begin{itemize}
\item kitka $F_\mu=\mu N$
\item paino $G=mg$, missä $g=9.81 m/s^2$
\end{itemize}

\end{document}