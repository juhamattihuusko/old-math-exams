\documentclass[12pt]{article}
\usepackage{graphicx} % Required for inserting images
\usepackage{amsmath,amssymb,amsthm,amsfonts}
\usepackage{xcolor}
\usepackage{tasks}
\usepackage{enumitem}
\usepackage[margin=2cm]{geometry}

\title{OAMK}
\author{Juha-Matti Huusko}
\date{August 2023}

\newcommand{\ratkaisu}[1]{{\color{blue}\quad\textrm{Solution } #1}}

\renewcommand{\ratkaisu}[1]{}

\newcommand{\yes}[0]{yes}
\newcommand{\no}[0]{no}

\renewcommand{\yes}[0]{}
\renewcommand{\no}[0]{}

\begin{document}
%\thispagestyle{empty}

%\section*{Uusintakoe 17.11.2023}
%\section*{Loppukoe 20.10.2023}
\subsection*{Applied Mathematics and Physics in Programming ID00CS50-3004}
\textbf{Mathematics, teacher:} Juha-Matti Huusko, juha-matti.huusko@oamk.fi

\noindent Answer to all six questions. In the end, there are some formulas.

$$
***
$$

\begin{enumerate}
\item 
Give an example of each type.
\begin{enumerate}
\item Differential equation with order $3$.\ratkaisu{K, N}
\item Differential equation which is separable and non-linear.\ratkaisu{K,M}
\item Differential equation which is linear and homogeneous.\ratkaisu{L,N}
\item Differential equation which is and non-homogeneous.\ratkaisu{None}
\end{enumerate}

\item Show that the functions $y$ are solutions to the corresponding differential equations.
\begin{enumerate}
\item Show that $y=\dfrac{1}{(1-x)^3}$ is not a particular solution for $y'=2(1-x)y^2$.
\item Show that $y=\dfrac{e^{-2x}}{3}$ is a particular solution for $y'+5y=e^{-2x}$.
\end{enumerate}
\ratkaisu{(a) The derivative is
$$
y'=\frac{d}{dx}(1-x)^{-2}=-2\cdot (1-x)^{-2-1}\frac{d}{dx}(1-x)=-2\cdot (1-x)^{-3}\cdot(-1)=\frac{2}{(1-x)^3}.
$$
Moreover
$$
2(1-x)y^2=\frac{2}{(1-x)^3}=y'.
$$
(b) We have
$$
y'+5y=\frac{-2}{3}e^{-2x}+\frac{5}{3}e^{-2x}
=\frac{3}{3}e^{-2x}=e^{-2x}.
$$
}

\item The general solution of $y'=4x^2$ is $y=\frac{4}{3}x^3+C$, where $C$ is any constant. Which particular solution passes through the point $(-3,-40)$?
%(that is, satisfies $x=-3$, $y=-30$)?
\ratkaisu{We have $y(x)=y=\frac{4}{3}x^3+C.$ Set $x=-3$ and $y(-3)=-40$ to obtain
$$
-40=\frac{4}{3}(-3)^3+C,
$$
that is
$$
-40=-36+C.
$$
We have $C=-4$. The desired solution is $y(x)=y=\frac{4}{3}x^3-4$.
}
\newpage

\item The solution of
$$
y'+p(x)y=q(x)
$$
is given by the formula
$$
y(x)=\frac{C}{\mu(x)}+\frac{1}{\mu(x)}\int \mu(x)q(x)dx,\quad\textrm{where}\quad
\mu(x)=e^{\int p(x)dx}.
$$
Use the formula to solve
$$
y'+\frac{5}{x}y=x^2
$$
by following the instructions.
\begin{enumerate}
\item Identify $p(x)$ and $q(x)$.\ratkaisu{$p(x)=-\frac{5}{x}$ and $q(x)=x^2$}
\item Calculate $\int p(x)dx$. Don't add a constant $C$ yet.\ratkaisu{$\int p(x)dx=-5\ln(x)$}
\item Simplify $\mu(x)=e^{\int p(x)dx}$ and $\frac{1}{\mu(x)}$\ratkaisu{$\mu(x)=\dfrac{1}{x^5}$ and $\frac{1}{\mu(x)}=x^5$}
\item Calculate $\int \mu(x)q(x)dx$.
\ratkaisu{$-\frac{1}{2x^2}$}
\item The solution is $y(x)=\frac{C}{\mu(x)}+\frac{1}{\mu(x)}\int \mu(x)q(x)dx$.
\ratkaisu{$y(x)=Cx^5-\frac12 x^3$}
\end{enumerate}

\item Consider the $2\pi$ periodic function $f(x)$ which satisfies
$$
f(t)
=\begin{cases}
x,&\textrm{ if }0\leq x\leq \pi\\
-x,&\textrm{ if }-\pi\leq x\leq 0.
\end{cases}
$$
Which of the Fourier coefficients $a_0$, $a_1$, $a_2$, $b_1$ and $b_2$ are nonzero?\ratkaisu{Because $f(-x)=-f(x)$, the function $f$ is odd and the list of formulas gives
$$
f(x)
=\sum_{n=1}^\infty b_n\sin(nx),
\quad\textrm{where}\quad
b_n=\frac{2}{\pi}\int_{0}^{\pi}
f(x)\sin(nx)dx
$$
Therefore, $a_0=a_1=a_2=0$. Moreover,
$$
b_1=\frac{2}{\pi}\int_{0}^{\pi}
\sin(x)dx
=\frac{2}{\pi}(\cos(0)-\cos(\pi))=\frac{4}{\pi},
$$
which is nonzero. Moreover,
$$
b_2=\frac{2}{\pi}\int_{0}^{\pi}
\sin(2x)dx
=\frac{2}{\pi}\frac{1}{2}(\cos(2\cdot 0)-\cos(2\cdot \pi))=0.
$$
\textbf{In conclusion, the only nonzero coefficient is $b_1$.}
}

\item Find the discrete Fourier transform of $[2+3i,3,-5,7]$. In other words, calculate by hand $\texttt{fft}([2+3i,3,-5,7])$.
\ratkaisu{By the formula, we obtain
$$
\begin{cases}
y_0&=2+3-5+7=7\\
y_1&=2-3i-(-5)+7i=7+4i\\
y_2&=2-3+(-5)-7=13\\
y_3&=2+3i-(-5)-7i=7-4i.
\end{cases}
$$
That is, $\texttt{fft}([2,3])=[7,7+4i,-13,7-4i]$}
\end{enumerate}

\textbf{In the following pages, there are some formulas.}

\newpage
\section*{Formulas}
\subsection*{Differentiation and integration}

$$
\begin{array}{rl|rl}
\textbf{Differentiation} && \textbf{Integration}&\\[2mm]
Dx^n&=nx^{n-1}     \qquad\qquad&\qquad\qquad\int x^ndx&=\frac{x^{n+1}}{n+1}+C \\[2mm]
De^x&=e^x &\int e^xdx&=e^x+C\\[2mm]
Db^x&=b^x\ln(b) & \int b^xdx&=\frac{b^x}{\ln(b)}\\[2mm]
D\ln(x)&=\frac{1}{x} &&\\[2mm]
D\ln|x|&=\frac{1}{x} &\int\frac{1}{x}dx&=\ln|x|+C\\[2mm]
D\log_a(x)&=\frac{1}{x\ln(a)} &&\\[2mm]
D\log_a|x|&=\frac{1}{x\ln(a)} &&\\[2mm]
D\sin(x)&=\cos(x)   &\int\cos(x)dx&=\sin(x)+C\\[2mm]
D\cos(x)&=-\sin(x)  &\int\sin(x)dx&=-\cos(x)+C\\[2mm]
D\tan(x)&=1+\tan^2(x) \qquad&\qquad\int 1+\tan^2(x)dx&=\tan(x)+C\\[2mm]

Dx\ln(x)-x&=\ln(x) & \int\ln(x)dx&=x\ln(x)-x+C\\[10mm]

D\arcsin(x)&=\frac{1}{\sqrt{1-x^2}} & \int\frac{1}{\sqrt{1-x^2}}&=\arcsin(x)+C\\
D\arccos(x)&=\frac{1}{-\sqrt{1-x^2}} & \int\frac{1}{-\sqrt{1-x^2}}&=\arccos(x)+C\\
D\arctan(x)&=\frac{1}{1+x^2} & \int\frac{1}{1+x^2}&=\arctan(x)+C\\

D\sinh(x)&=\cosh(x) &&\\
D\cosh(x)&=\sinh(x) &&\\
D\tanh(x)&=\frac{1}{\cosh^2(x)} &&\\
\end{array}  
$$
\vspace{1cm}
$$
\begin{array}{rl|rl}
\textbf{Differentiation} && \textbf{Integration}&\\[2mm]
D f(g(x))&=f'(g(x))g'(x) & \int f'(g(x))g'(x)dx&=f(g(x))+C\\[2mm]
\textrm{Special cases} &&&\\
D\ln(g(x))&=\frac{g'(x)}{g(x)} & \int \frac{g'(x)}{g(x)}dx&=\ln(g(x))+C\\[2mm]
D e^{g(x)}&=e^{g(x)}g'(x) & \int g'(x)e^{g(x)}dx&=e^{g(x)}+C\\[10mm]
D fg&=f'g+fg'& \int f'g dx&=fg-\int fg'dx\\[2mm]
D (f/g)&=(gf'-fg')/g^2 &&\\[2mm]
D f^{-1}(x)&=\frac{1}{f'(f^{-1}(x))}&&\\
\end{array}  
$$

\newpage
\subsection*{Solution formula}

The solution of
$$
y'+p(x)y=q(x)
$$
is
$$
y(x)=\frac{C}{\mu(x)}+\frac{1}{\mu(x)}\int \mu(x)q(x)dx,\quad\textrm{where}\quad
\mu(x)=e^{\int p(x)dx}.
$$

\subsection*{Fourier series}

If $f$ is periodic with period $2\pi$ and $f$, $f'$ and $f''$ are piece-wise continuous, then
$$
f(x)
=\frac{a_0}{2}
+\sum_{n=1}^\infty a_n\cos(nx)+b_n\sin(nx),
$$
where
\begin{equation*}
\begin{split}
a_0&=\frac{1}{\pi}\int_{-\pi}^{\pi}
f(x)dx\\
a_n&=\frac{1}{\pi}\int_{-\pi}^{\pi}
f(x)\cos(nx)dx\\
b_n&=\frac{1}{\pi}\int_{-\pi}^{\pi}
f(x)\sin(nx)dx\\
\end{split}
\end{equation*}
Moreover, if $f$ is odd, that is, $f(-x)=-f(x)$,
then
$$
f(x)
=\sum_{n=1}^\infty b_n\sin(nx),
\quad\textrm{where}\quad
b_n=\frac{2}{\pi}\int_{0}^{\pi}
f(x)\sin(nx)dx,
$$
and if $f$ is even, that is, $f(-x)=f(x)$, then
$$
f(x)
=\frac{a_0}{2}
+\sum_{n=1}^\infty a_n\cos(nx)
\quad\textrm{where}\quad
a_n=\frac{2}{\pi}\int_{0}^{\pi}
f(x)\cos(nx)dx.
$$

\subsection*{Discrete Fourier transform / FFT}

Transform and inverse transform
$$
\begin{cases}
y_0&=x_0+x_1\\
y_1&=x_0-x_1
\end{cases},\quad
\begin{cases}
y_0&=\frac{1}{2}(x_0+x_1)\\
y_1&=\frac{1}{2}(x_0-x_1)
\end{cases}
$$
Transform and inverse transform
$$
\begin{cases}
y_0&=x_0+x_1+x_2+x_3\\
y_1&=x_0-ix_1-x_2+ix_3\\
y_2&=x_0-x_1+x_2-x_3\\
y_3&=x_0+ix_1-x_2-ix_3\\
\end{cases},\quad
\begin{cases}
y_0&=\frac{1}{4}(x_0+x_1+x_2+x_3)\\
y_1&=\frac{1}{4}(x_0+ix_1-x_2-ix_3)\\
y_2&=\frac{1}{4}(x_0-x_1+x_2-x_3)\\
y_3&=\frac{1}{4}(x_0-ix_1-x_2+ix_3)\\
\end{cases}
$$

\newpage


\end{document}
