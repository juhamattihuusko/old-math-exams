\documentclass[12pt]{article}
\usepackage{graphicx} % Required for inserting images
\usepackage{amsmath,amssymb,amsthm,amsfonts}
\usepackage{xcolor}

\usepackage[margin=1cm]{geometry}

\title{OAMK}
\author{Juha-Matti Huusko}
\date{August 2023}

%\renewcommand{\ratkaisu}[1]{}

\newcommand{\hatu}{\hat{u}}
\newcommand{\hatv}{\hat{v}}
\newcommand{\hatw}{\hat{w}}
\newcommand{\hatn}{\hat{n}}


\newcommand{\vu}{\overline{u}}
\newcommand{\vv}{\overline{v}}
\newcommand{\vw}{\overline{w}}
\newcommand{\vp}{\overline{p}}
\newcommand{\vn}{\overline{n}}

\newcommand{\va}{\overline{a}}
\newcommand{\vb}{\overline{b}}
\newcommand{\vc}{\overline{c}}

\newcommand{\vi}{\hat{\imath}}
\newcommand{\vj}{\hat{\jmath}}
\newcommand{\vk}{\hat{k}}

\newcommand{\ratkaisu}[1]{\hfill{\color{blue}\quad\textrm{Ratkaisu: } #1}}

\newcommand{\ratkaisuu}[1]{{\color{blue}\textrm{Ratkaisu: } #1}}

\renewcommand{\ratkaisu}[1]{}
\renewcommand{\ratkaisuu}[1]{}

\newcommand{\vihje}[1]{{\color{red}Vihje. #1}}
\newcommand{\extra}[0]{\textbf{Extra.}~}

\begin{document}
\thispagestyle{empty}

\section*{Koe 17.1.2024}
%\section*{Loppukoe 20.10.2023}
\paragraph*{Tietotekniikan sovellusprojekti IN00ED23-3001,\\ matematiikan osio, Vektorit ja kompleksilukujen perusteet (3op)}

\begin{enumerate}
%\setlength{\itemsep}{40pt}
\item Laske vektorien $\vu=(1,12,-2)$ ja $\vv=(3,0,1)$ pistetulo.\ratkaisu{$1$}

%\item Laske vektorin $\vu=(1,5,-1)$ pituus.\ratkaisu{$\sqrt{27}$}

\item Määritä vektorin $\vu=2\vi-4\vj+4\vk$ suuntainen yksikkövektori.\ratkaisu{$\hat{u}=\frac{1}{3}\vi-\frac{2}{3}\vj+\frac{2}{3}\vk$}

\item (Muokattu.) Laske vektorien $(1,2,2)$ ja $(-1,2,2)$ välinen kulma.\ratkaisu{noin $63^\circ$}

%\item Laske vektorien $\vu=(1,2,3)$ ja $\vw=(-7,2,1)$ välinen kulma.\ratkaisu{$90^\circ$}

\item Laske tetraedrin tilavuus, kun sen kärjet ovat pisteissä $A=(2,1,-1)$, $B=(1,0,2)$, $C=(-1,-2,2)$ ja $D=(-1,2,-1)$.
%\vihje{Tilavuuden voi laskea kaavalla $V=\pm\frac{1}{6}(\vu\times\vv)\cdot\vw$, missä vektorit $\vu$, $\vv$ ja $\vw$ ovat tetraedrin kolmen särmää.}\\
\ratkaisuu{$(\vu\times\vv)\cdot\vw=-24$, joten $V=4$}

\item Määritä pisteen $P=(2,2)$ etäisyys pisteiden $A=(1,-1)$ ja $B=(5,2)$ kautta kulkevasta suorasta.\\
\emph{Vihje. Jos $\vn$ on suoran jokin normaalivektori ja $\vv=P-A$, niin etäisyys on luku $\vv\cdot\hatn$.}\\
\ratkaisuu{suoran suuntavektori $\vu=(4,3)$, $\vn=(-3,4)$, $\hatn=(-0.6,0.8)$, $\vv=(1,3)$,\\ 
etäisyys on $(1,3)\cdot(-0.6,0.8)=-0.6+2.4=1.8$.}

\item (Muokattu.) Laske kompleksilukujen tulo
$$
(3-4i)(1+5i).
$$\ratkaisu{$-17 + 19i$}

\item Kirjoita kompleksiluku $z=5e^{i\pi/3}$ summamuodossa $z=x+iy$. (Luvuille $x$ ja $y$ riittää likiarvo.) \ratkaisu{$z\approx 2.50 + 4.33i$}

\item Etsi toisen asteen yhtälön
$$
x^2-2x+2=0.
$$
kompleksiset ratkaisut.\ratkaisu{$x=1+i$ \textrm{ tai } $x=1-i$}
\end{enumerate}

\section*{Kaavoja (jatkuu kääntöpuolella)}

\begin{equation*}
\begin{split}
(u_1,u_2,u_3)\cdot(v_1,v_2,v_3)
&=u_1v_1+u_2v_2+u_3v_3,\qquad \frac{\pi}{3}=\frac{\pi}{3}\cdot\frac{180^\circ}{\pi}=60^\circ,\quad \vu=(a,b)\Rightarrow \vn=(b,-a)\quad \textrm{(Esim.)}\\
|(u_1,u_2,u_3)|&=\sqrt{u_1^2+u_2^2+u_3^2},
\quad \hatu=\frac{1}{|\vu|}\vu,
\qquad (u_1,u_2,u_3)\times(v_1,v_2,v_3)
=\begin{pmatrix}
u_2v_3-u_3v_2\\
u_3v_1-u_1v_3\\
u_1v_2-u_2v_1
\end{pmatrix}
\\
\cos\alpha&=\frac{\vu\cdot\vv}{|\vu||\vv|},
\quad \theta=\arctan(y/x)\\
V_S&=|(\vu\times\vv)\cdot\vw|,\quad
V_P=\frac{1}{2}|(\vu\times\vv)\cdot\vw|,\quad
V_T=\frac{1}{6}|(\vu\times\vv)\cdot\vw|,\quad
A=|\vu\times\vv|\\
\end{split}
\end{equation*}

\begin{equation*}
\begin{split}
&\sqrt{-a}=\sqrt{a}\sqrt{-1}=i\sqrt{a},\quad i^2=-1\\
&z^*=(x+iy)^*=x-iy,\quad |x+iy|=\sqrt{x^2+y^2}\\
&\theta=\arctan(y/x),\quad e^{i\theta}=\cos(\theta)+i\sin(\theta)\\
&\frac{z}{w}=\frac{zw^*}{|w|^2},\quad\textrm{eli}\quad
\frac{a+bi}{c+di}
=\frac{(a+bi)(c-di)}{(c+di)(c-di)}
=\frac{(a+bi)(c-di)}{c^2+d^2}
\end{split}
\end{equation*}

\begin{equation*}
\begin{split}
\frac{a}{b}\bigg/\frac{c}{d}
=\frac{a}{b}\cdot\frac{d}{c},\qquad
ax^2+bx+c=0\quad&\Leftrightarrow\quad x=\frac{-b\pm\sqrt{b^2-4ac}}{2a}\\[2mm]
\log_a(x)=y\quad&\Leftrightarrow\quad a^y=x
\end{split}
\end{equation*}


$$
\log_a(1)=0,\quad
\log_a(a)=1,\quad
\log_a(a^x)=x,\quad
a^{\log_a(x)}=x
$$

\begin{equation*}
\begin{split}
\log_a(b^c)&=c\log_a(b)\\
\log_a(xy)&=\log_a(x)+\log_a(y)\\
\log_a\left(\frac{x}{y}\right)
&=\log_a(x)-\log_a(y)\\
\log_a(x)&=\frac{\log_b(x)}{\log_b(a)}
\end{split}
\end{equation*}

$$
\textrm{lb}(x)=\log_2(x),\quad
\lg(x)=\log_{10}(x),\quad
\ln(x)=\log_e(x),\quad
e\approx 2,72
$$

\end{document}
