\documentclass[12pt]{article}
\usepackage{graphicx} % Required for inserting images
\usepackage{amsmath,amssymb,amsthm,amsfonts}
\usepackage{xcolor}
\usepackage{tasks}
\usepackage{enumitem}
\usepackage[margin=2cm]{geometry}
\usepackage{tkz-euclide}
\usetikzlibrary{calc}
\usepackage{tikz}
\usepackage{pgfplots}

\usepackage[utf8]{inputenc}
\usepackage[T1]{fontenc}
\usepackage{amsmath}
\usepackage{amsfonts}
\usepackage{amssymb}
\usepackage[version=4]{mhchem}
\usepackage{stmaryrd}
\usepackage{enumerate}
\usepackage{multicol}
\usepackage{xcolor}
\usepackage{graphicx}
\usepackage{ulem}
\usepackage{cancel}
\usepackage{tikz}
\usepackage{tkz-euclide}
\usepackage[finnish]{babel}

%\usepackage[style=alphabetic,]{biblatex}

%\usepackage[margin=2cm]{geometry}

\newcommand{\brac}[1]{\left(#1\right)}
\newcommand{\sqbrac}[1]{\left[#1\right]}
\newcommand{\set}[1]{\left\{#1\right\}}

\newcommand{\dd}[0]{\mathrm{d}}
\newcommand{\dx}[0]{\mathrm{d}x}

\newcommand{\hatu}{\hat{u}}
\newcommand{\hatv}{\hat{v}}
\newcommand{\hatw}{\hat{w}}
\newcommand{\hatn}{\hat{n}}

\newcommand{\vu}{\overline{u}}
\newcommand{\vv}{\overline{v}}
\newcommand{\vw}{\overline{w}}
\newcommand{\vp}{\overline{p}}
\newcommand{\vn}{\overline{n}}

\newcommand{\va}{\overline{a}}
\newcommand{\vb}{\overline{b}}
\newcommand{\vc}{\overline{c}}
\newcommand{\vd}{\overline{d}}


\newcommand{\vi}{\hat{\imath}}
\newcommand{\vj}{\hat{\jmath}}
\newcommand{\vk}{\hat{k}}

\newcommand{\ratkaisu}[1]{\hfill{\color{blue}\quad\textrm{Ratkaisu: } #1}}

\newcommand{\ratkaisuu}[1]{{\color{blue}\textrm{Ratkaisu: } #1}}

\newcommand{\kaava}[1]{{\color{green!50!black}#1}}

%\renewcommand{\ratkaisu}[1]{}
%\renewcommand{\ratkaisuu}[1]{}
%\renewcommand{\kaava}[1]{}

\newcommand{\vihje}[1]{{\color{red}Vihje. #1}}
\newcommand{\extra}[0]{\textbf{Extra.}~}

\title{OAMK}
\author{Juha-Matti Huusko}
\date{August 2023}

\renewcommand{\ratkaisu}[1]{{\color{blue}\quad\textrm{Ratkaisu: } #1}}

\renewcommand{\ratkaisu}[1]{}

\begin{document}

Tarkastellaan yhtälöä $x=1/\exp(x)$. Ratkaise $x$ yhden desimaalin tarkkuudella. Voit käyttää hyväksi Octavella generoitua taulukkoa

\begin{verbatim}
x=[1:16]'/10;
A=[x,1./exp(x)]
A =

   0.1000   0.9048
   0.2000   0.8187
   0.3000   0.7408
   0.4000   0.6703
   0.5000   0.6065
   0.6000   0.5488
   0.7000   0.4966
   0.8000   0.4493
   0.9000   0.4066
   1.0000   0.3679
   1.1000   0.3329
   1.2000   0.3012
   1.3000   0.2725
   1.4000   0.2466
   1.5000   0.2231
   1.6000   0.2019
\end{verbatim}

\textbf{Ratkaisu}

\begin{verbatim}
x=[1:16]'/10;
A=[x,1./exp(x)]
A =

   0.1000   0.9048
   0.2000   0.8187
   0.3000   0.7408
   0.4000   0.6703
   0.5000  < 0.6065
   0.6000  > 0.5488
   0.7000   0.4966
   0.8000   0.4493
   0.9000   0.4066
   1.0000   0.3679
   1.1000   0.3329
   1.2000   0.3012
   1.3000   0.2725
   1.4000   0.2466
   1.5000   0.2231
   1.6000   0.2019
\end{verbatim}

Siis $0.5<x<0.6$.

Tarkastellaan funktiota $f(x)=x^3/\exp(2x)+0.1$. Etsitään funktion maksimikohta $x_0$. Ratkaise $x_0$ yhden desimaalin tarkkuudella. Voit käyttää hyväksi Octavella generoitua taulukkoa.

\begin{verbatim}
x=[1:16]'/10;
fx=x.^3./exp(2*x)+0.1;
A=[x,fx]
A =

   0.1000   0.1008
   0.2000   0.1054
   0.3000   0.1148
   0.4000   0.1288
   0.5000   0.1460
   0.6000   0.1651
   0.7000   0.1846
   0.8000   0.2034
   0.9000   0.2205
   1.0000   0.2353
   1.1000   0.2475
   1.2000   0.2568
   1.3000   0.2632
   1.4000   0.2669
   1.5000   0.2680
   1.6000   0.2670
\end{verbatim}

Ratkaisu

\begin{verbatim}
x=[1:16]'/10;
fx=x.^3./exp(2*x)+0.1;
A=[x,fx]
A =

   0.1000   0.1008
   0.2000   0.1054
   0.3000   0.1148
   0.4000   0.1288
   0.5000   0.1460
   0.6000   0.1651
   0.7000   0.1846
   0.8000   0.2034
   0.9000   0.2205
   1.0000   0.2353
   1.1000   0.2475
   1.2000   0.2568
   1.3000   0.2632
   1.4000   0.2669
   1.5000   0.2680 <-max
   1.6000   0.2670
\end{verbatim}

Maksimikohdalle pätee $1.4<x_0<16$.



\end{document}