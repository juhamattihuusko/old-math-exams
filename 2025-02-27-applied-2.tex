\documentclass[12pt]{article}
\usepackage{graphicx} % Required for inserting images
\usepackage{amsmath,amssymb,amsthm,amsfonts}
\usepackage{xcolor}
\usepackage{tasks}
\usepackage{enumitem}
\usepackage[margin=2cm]{geometry}

\title{OAMK}
\author{Juha-Matti Huusko}
\date{August 2023}

\newcommand{\ratkaisu}[1]{{\color{blue}\quad\textrm{Solution } #1}}

\renewcommand{\ratkaisu}[1]{}

\begin{document}
\thispagestyle{empty}

\section*{Exam 27.2.2025}
%\section*{Uusintakoe 17.11.2023}
%\section*{Loppukoe 20.10.2023}
\subsection*{Applied Mathematics and Physics in Programming ID00CS50-3003}

\begin{enumerate}
\item Match each of the differential equations
\begin{enumerate}[label=(\roman*)]
\item [(E1)] $y'''+4xy'+4y=0$ %ORDER 3
\item [(E2)] $y''+4y=0$ %CONSTANT COEFF
\item [(E3)] $y'-\tan(x)y=0$ %SEP
\item [(E4)] $y''+x\sin(y)=0$ %NONLIN
\item [(E5)] $y''+xy'+4y=x^2$ %NON-HOMO
\end{enumerate}
with one of the following properties. (One equation for one property. No need to justify the answer.)
\begin{enumerate}
\item Which equation is of order 3?
\item Which equation is separable?
\item Which equation has constant coefficients?
\item Which equation is nonlinear?
\item Which equation is linear and non-homogeneous?
\end{enumerate}

\item Match each of the differential equations
\begin{enumerate}
\item [(F1)] $my''=-mg$ %falling
\item [(F2)] $y'+ay=0$ %radioactive
\item [(F3)] $my''+ky=0$ %harmonic
\item [(F4)] $my''+Ry'+my=0$ %harmonic with damping
\item [(F5)]  $y''+\sin(y)=0$ %pendulum
\end{enumerate}
with one of the following physics phenomena. (One equation for one phenomenon. No need to justify the answer.)

\begin{enumerate}
\item Which equation is about free fall?
\item Which equation is about radioactive decay?
\item Which equation is about harmonic oscillator (mass and spring) without damping?
\item Which equation is about harmonic oscillator with damping?
\item Which equation is about pendulum?
\end{enumerate}

\item Find the general solution of $y'=6x+2$.

Find the particular solution which satisfies the initial condition $y(0)=3$. For this particular solution, find $y(2)$.

\item A mass ($m=1~kg$) lies on a frictionless rail. The mass is connected to the end of the rail with a spring ($k=25~kg/s^2$). The position of the mass at time t is described by the function
$x(t)=A\cos(\omega t)+B\sin(\omega t)$.

A person pulls the mass $0.25~m$ from the neutral position. At the moment $t=0$, we have $x(0)=0.25$ and $x'(0)=0$. The person releases the mass and the mass will oscillate back and forth.

\textbf{Tasks.}
\begin{enumerate}
\item Express $\omega$ in terms of $k$ and $m$. %OMEGA=SQRT(k/m)
\item When does the mass for the first time return to the starting position?% (x(t)=x(0)) => t=4pi
\item What is the maximum velocity of the mass?% => 0.125m/s
\end{enumerate}

\item Consider the equation
$$
y'-2\tan(x)y=x.
$$
Solve it, for example, by following the instructions.
\begin{enumerate}
\item Identify $p(x)$ and $q(x)$.\ratkaisu{$p(x)=\frac{1}{x}$ and $q(x)=x^2$}
\item Calculate $\int p(x)dx$. Don't add a constant $C$ yet.\ratkaisu{$\int p(x)dx=\ln(x)$}
\item Simplify $\mu(x)=e^{\int p(x)dx}$ and $\frac{1}{\mu(x)}$\ratkaisu{$\mu(x)=x$ and $\frac{1}{\mu(x)}=\frac{1}{x}$}
\item Calculate $\int \mu(x)q(x)dx$.
\ratkaisu{$\frac{x^4}{4}$}
\item The solution is $y(x)=\frac{C}{\mu(x)}+\frac{1}{\mu(x)}\int \mu(x)q(x)dx$.
\ratkaisu{$y(x)=\frac{C}{x}+\frac{x^3}{4}$}
\end{enumerate}

\item Consider the $2\pi$ periodic function $f$ which satisfies $f(x)=|x|$ for $-\pi<x<\pi$.

Which of the Fourier coefficients $a_0$, $a_1$, $a_2$, $b_1$, $b_2$ are zero?
\end{enumerate}

%Because the function is even, we have b_1=0 and b_2=0.

%a_0 =/= 0
%a_1 =/= 0
%a_2 =0
%Moreover, a_0=2\int_0^\pi xdx=pi^2

%a_1=-4
%a_2=0
%a_2,b1,b2 <=

\newpage
\section*{Formulas}
\subsection*{Differentiation and integration}

$$
\begin{array}{rl|rl}
\textbf{Differentiation} && \textbf{Integration}&\\[2mm]
Dx^n&=nx^{n-1}     \qquad\qquad&\qquad\qquad\int x^ndx&=\frac{x^{n+1}}{n+1}+C \\[2mm]
De^x&=e^x &\int e^xdx&=e^x+C\\[2mm]
Db^x&=b^x\ln(b) & \int b^xdx&=\frac{b^x}{\ln(b)}\\[2mm]
D\ln(x)&=\frac{1}{x} &&\\[2mm]
D\ln|x|&=\frac{1}{x} &\int\frac{1}{x}dx&=\ln|x|+C\\[2mm]
D\log_a(x)&=\frac{1}{x\ln(a)} &&\\[2mm]
D\log_a|x|&=\frac{1}{x\ln(a)} &&\\[2mm]
D\sin(x)&=\cos(x)   &\int\cos(x)dx&=\sin(x)+C\\[2mm]
D\cos(x)&=-\sin(x)  &\int\sin(x)dx&=-\cos(x)+C\\[2mm]
D\tan(x)&=1+\tan^2(x) \qquad&\qquad\int 1+\tan^2(x)dx&=\tan(x)+C\\[2mm]

Dx\ln(x)-x&=\ln(x) & \int\ln(x)dx&=x\ln(x)-x+C\\[10mm]

D\arcsin(x)&=\frac{1}{\sqrt{1-x^2}} & \int\frac{1}{\sqrt{1-x^2}}&=\arcsin(x)+C\\
D\arccos(x)&=\frac{1}{-\sqrt{1-x^2}} & \int\frac{1}{-\sqrt{1-x^2}}&=\arccos(x)+C\\
D\arctan(x)&=\frac{1}{1+x^2} & \int\frac{1}{1+x^2}&=\arctan(x)+C\\

D\sinh(x)&=\cosh(x) &&\\
D\cosh(x)&=\sinh(x) &&\\
D\tanh(x)&=\frac{1}{\cosh^2(x)} &&\\
\end{array}  
$$
\vspace{1cm}
$$
\begin{array}{rl|rl}
\textbf{Differentiation} && \textbf{Integration}&\\[2mm]
D f(g(x))&=f'(g(x))g'(x) & \int f(g(x))g'(x)dx&=f(g(x))+C\\[2mm]
\textrm{Special cases} &&&\\
D\ln(g(x))&=\frac{g'(x)}{g(x)} & \int \frac{g'(x)}{g(x)}dx&=\ln(g(x))+C\\[2mm]
D e^{g(x)}&=e^{g(x)}g'(x) & \int g'(x)e^{g(x)}dx&=e^{g(x)}+C\\[10mm]
D fg&=f'g+fg'& \int f'g dx&=fg-\int fg'dx\\[2mm]
D (f/g)&=(gf'-fg')/g^2 &&\\[2mm]
\end{array}  
$$

\newpage

\subsection*{Physics}

\subsubsection*{Newtonian mechanics}

Some forces
\begin{itemize}
\item $G=mg$ (gravitation)
\item $F=kx$ (spring)
\item $F=av^2$ (air resistance)
\end{itemize}
Differential equation is given by Newton's second law
$$
F_{\textrm{total}}=mx''.
$$

\subsubsection*{Lagrangian mechanics}
Lagrange function is the difference of kinetic energy and potential energy. As a formula, $L=T-U$ or $L=K-P$
\begin{itemize}
\item Object in free fall $L=\frac12 my''-mgy$
\item Pendulum $L=\frac12 mL^2\theta'(t)-mgL(1-\cos(\theta(t)))$
\item Mass and spring $L=\frac12 mx'(t)^2-\frac12 kx(t)^2$
\end{itemize}
Differential equation is given by Euler-Lagrange equation
$$
\frac{d}{dt}\left(\frac{\partial L}{\partial x'}\right)-\frac{\partial L}{\partial x}=0.
$$

\subsubsection*{Other differential equations}

\begin{itemize}
\item Radioactive decay $N'(t)=-aN(t)$
\item Newton's law of cooling $T'(t)=-k(T(t)-T_a)$
\end{itemize}

\section*{Trigonometry}

\begin{equation*}
\begin{split}
\sin(a+b)&=\sin(a)\cos(b)+\cos(a)\sin(b)\\
\cos(a+b)&=\cos(a)\cos(b)-\sin(a)\sin(b)\\
2\sin(a)\sin(b) &= \cos(a-b)-\cos(a+b)\\
2\sin(a)\cos(b) &= \sin(a-b)+\sin(a+b)\\
2\cos(a)\cos(b) &= \cos(a-b)+\cos(a+b)\\
\end{split}
\end{equation*}

\newpage
\subsection*{Differential equations}
\subsubsection*{Second order linear ODE with constant coefficients}

\begin{itemize}
\item ODE $y''+by'+cy=0$
\item Characteristic equation $r^2+br+c=0$
\end{itemize}
Cases
\begin{itemize}
\item $r_1,r_2\in\mathbb{R}$ solution
$$
y(x)=A\exp(r_1x)+B\exp(r_2x)
$$
\item $r_1=r_2=r$ solution
$$
y(x)=A\exp(rx)+Bx\exp(rx)
$$
\item $r_1=a+bi$ solution
$$
y(x)=\exp(ax)(A\cos(bx)+B\sin(bx))
$$
\end{itemize}

\subsubsection*{Integrable ODE}
The solution of
$$
y'=q(x)
$$
is $y(x)=\int q(x)dx$

\subsubsection*{Separable ODE}
If you can arrange the equation as $$a(y)dy=b(x)dx,$$
then you can integrate to obtain $$\int a(y)dy=\int b(x)dx.$$

\subsubsection*{First order linear ODE}
The solution of
$$
y'+p(x)y=q(x)
$$
is
$$
y(x)=\frac{C}{\mu(x)}+\frac{1}{\mu(x)}\int \mu(x)q(x)dx,\quad\textrm{where}\quad
\mu(x)=e^{\int p(x)dx}.
$$

\newpage
\subsection*{Fourier series}

If $f$ is periodic with period $2\pi$ and $f$, $f'$ and $f''$ are piece-wise continuous, then
$$
f(x)
=\frac{a_0}{2}
+\sum_{n=1}^\infty a_n\cos(nx)+b_n\sin(nx),
$$
where
\begin{equation*}
\begin{split}
a_0&=\frac{1}{\pi}\int_{-\pi}^{\pi}
f(x)dx\\
a_n&=\frac{1}{\pi}\int_{-\pi}^{\pi}
f(x)\cos(nx)dx\\
b_n&=\frac{1}{\pi}\int_{-\pi}^{\pi}
f(x)\sin(nx)dx\\
\end{split}
\end{equation*}
Moreover, if $f$ is odd, that is, $f(-x)=-f(x)$,
then
$$
f(x)
=\sum_{n=1}^\infty b_n\sin(nx),
$$
and if $f$ is even, that is, $f(-x)=f(x)$, then
$$
f(x)
=\frac{a_0}{2}
+\sum_{n=1}^\infty a_n\cos(nx)
$$

\subsection*{Discrete Fourier transform / FFT}

Transform and inverse transform
$$
\begin{cases}
y_0&=x_0+x_1\\
y_1&=x_0-x_1
\end{cases},\quad
\begin{cases}
y_0&=\frac{1}{2}(x_0+x_1)\\
y_1&=\frac{1}{2}(x_0-x_1)
\end{cases}
$$
Transform and inverse transform
$$
\begin{cases}
y_0&=x_0+x_1+x_2+x_3\\
y_1&=x_0-ix_1-x_2+ix_3\\
y_2&=x_0-x_1+x_2-x_3\\
y_3&=x_0+ix_1-x_2-ix_3\\
\end{cases},\quad
\begin{cases}
y_0&=\frac{1}{4}(x_0+x_1+x_2+x_3)\\
y_1&=\frac{1}{4}(x_0+ix_1-x_2-ix_3)\\
y_2&=\frac{1}{4}(x_0-x_1+x_2-x_3)\\
y_3&=\frac{1}{4}(x_0-ix_1-x_2+ix_3)\\
\end{cases}
$$


\end{document}
