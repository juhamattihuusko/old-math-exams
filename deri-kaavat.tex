\documentclass[12pt]{article}
\usepackage{graphicx} % Required for inserting images
\usepackage{amsmath,amssymb,amsthm,amsfonts}
\usepackage{xcolor}

\usepackage[margin=1cm]{geometry}

\title{OAMK}
\author{Juha-Matti Huusko}
\date{August 2023}

%\renewcommand{\ratkaisu}[1]{}

\newcommand{\hatu}{\hat{u}}
\newcommand{\hatv}{\hat{v}}
\newcommand{\hatw}{\hat{w}}
\newcommand{\hatn}{\hat{n}}


\newcommand{\vu}{\overline{u}}
\newcommand{\vv}{\overline{v}}
\newcommand{\vw}{\overline{w}}
\newcommand{\vp}{\overline{p}}
\newcommand{\vn}{\overline{n}}

\newcommand{\va}{\overline{a}}
\newcommand{\vb}{\overline{b}}
\newcommand{\vc}{\overline{c}}

\newcommand{\vi}{\hat{\imath}}
\newcommand{\vj}{\hat{\jmath}}
\newcommand{\vk}{\hat{k}}

\newcommand{\ratkaisu}[1]{\hfill{\color{blue}\quad\textrm{Ratkaisu: } #1}}

\newcommand{\ratkaisuu}[1]{{\color{blue}\textrm{Ratkaisu: } #1}}

\renewcommand{\ratkaisu}[1]{}
\renewcommand{\ratkaisuu}[1]{}

\newcommand{\vihje}[1]{{\color{red}Vihje. #1}}
\newcommand{\extra}[0]{\textbf{Extra.}~}

\begin{document}
\thispagestyle{empty}

\section*{Koe 17.1.2024}
%\section*{Loppukoe 20.10.2023}
\paragraph*{Tietotekniikan sovellusprojekti IN00ED23-3001,\\ matematiikan osio, Vektorit ja kompleksilukujen perusteet (3op)}

Derivaatan johtolaskuissa muutoksen kerroin antaa derivaatan lausekkeen. Esimerkiksi
$$
(x+h)^2=x^2+\underbrace{2x}_{=D x^2}h+h^2.
$$
Nimittäin
$$
D x^2=\lim_{h\to 0}\frac{(x+h)^2-x^2}{h}
=\lim_{h\to 0}\frac{x^2+2xh+h^2-x^2}{h}
=\lim_{h\to 0}2x+h=2x.
$$
Termi, jossa ei ollut yhtään muuttujaa $h$ kumoutui, ja termi, jossa oli $h^2$ tai $h^3$ jne. meni nollaan raja-arvoa ottaessa.

Saman asian kertoo lineaarisen approksimaation kaava
$$
f(x+h)=f(x)+f'(x)h+h\underbrace{\varepsilon(h)},
$$
missä virhe $\varepsilon(h)\to 0$, kun $h\to 0$.

\section{Derivointikaavojen johtoa}

$$
\sin(x+h)=\sin(x)\underbrace{\cos(h)}_{\approx 1-h^2/2}
+\cos(x)\underbrace{\sin(h)}_{\approx h}
$$
antaa $D\sin(x)=\cos(x)$

$$
\cos(x+h)=\cos(x)\underbrace{\cos(h)}_{\approx 1-h^2/2}-\sin(x)\underbrace{\sin(h)}_{\approx h}
$$
antaa $D\cos(x)=-\sin(x)$

Jos $f(x)=C$ on vakiofunktio, niin
$$
f(x+h)=f(x)
$$
antaa $f'(x)=0$.

Jos $f(x)=ax$, missä $a$ on vakio, niin
$$
f(x+h)=a(x+h)=ax+ah
$$
antaa $f'(x)=a$.

Jos $f(x)=x^2$, niin
$$
(x+h)^2=x^2+2xh+h^2
$$
antaa $f'(x)=2x$.

Jos $f(x)=x^3$, niin
$$
(x+h)^3=x^3+3x^2h+3xh^2+1
$$
antaa $D x^3=3x^2$.

Jos $f(x)=x^n$, niin
$$
(x+h)^n=x^n+nx^{n-1}h+h^2(\ldots)
$$
antaa $D x^n=nx^{x-1}$.

Jos $f(x)=ag(x)$, niin
$$
f(x+h)=ag(x+h)=ag(x)+ag'(x)h+ah\varepsilon(h)
$$
antaa $f'(x)=ag'(x)$.

Jos $f(x)=g(ax)$, niin
$$
f(x+h)=g(a(x+h))=g(ax+ah)
=g(ax)+g'(ax)ah+ah\varepsilon(ah)
$$
antaa $Dg(ax)=ag'(x)$

Yhdistetylle funktiolle $f(g(x))$
$$
\frac{f(g(x+h))-f(g(x))}{h}
=\frac{f(g(x+h))-f(g(x))}{g(x+h)-g(x)}
\cdot\frac{g(x+h)-g(x)}{h}
$$
antaa $Df(g(x))=f'(g(x))g'(x)$.

Funktiolle $g(x)$ ja käänteisfunktiolle $g^{-1}(x)=f(x)$ pätee 
$$
f(g(x))=x
$$
puolittain derivoimalla
$$
f'(g(x))g'(x)=1
$$
jakamalla luvulla $f'(g(x))$
$$
g'(x)=\frac{1}{f'(g(x))}
$$
Siis
$$
(f^{-1})'(x)=\frac{1}{f'(f^{-1}(x))}
$$

Jos $a>0$, niin funktiolle $f(x)=a^x$ kaava
$$
\frac{a^{x+h}-a^x}{h}
=\frac{a^{x}a^{0+h}-a^{x}a^{0}}{h}
=a^x\frac{a^{0+h-a^0}}{h}
$$
antaa $f'(x)=f(x)\cdot f'(0)$.

\subsection{Luvun $e$ määritelmä}

Saatiin, että funktiolle $f(x)=a^x$ pätee $f'(x)=f(x)\cdot f'(0)$. Luku $f'(0)$ riippuu vain kantaluvusta $a$. Mitä suurempi $a$, sitä suurempi $f'(0)$.

Asetetaan, että $e$ on se luku, jolle $f'(0)=1$. Saadaan $De^x=e^x$.

Funktiolle $f(x)=e^x$ käänteisfunktio on $f^{-1}(x)=\ln(x)$, sen derivaatta
$$
D\ln(x)=\frac{1}{e^{\ln(x)}}=\frac{1}{x}.
$$
Kaava $D\ln(x)=\frac{1}{x}$ on todella yksinkertainen.

Koska integrointi on derivoinnin käänteisoperaatio, seuraa
$$
\int_1^e\frac{1}{x}dx=\ln(e)-\ln(1)=1-0=1.
$$
(Manne pitää tästä luvun $e$ määritelmästä.)

Luvulle $e$ nähtiin hetkessä kolme hienoa ominaisuutta. Sen vuoksi $e$ on kätevä ja luonnollinen monessa yhteydessä. Hienoja kaavoja löytyy paljon lisää.



\end{document}
